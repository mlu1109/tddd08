% This is a summary of the keywords given for the course TDDD08.
% There is no guarantee that ANYTHING in this text is correct.

\documentclass{article}
% Följande rad ska göra det möjligt att använda svenska bokstäver, som å, ä, ö. Kravet är 
% då att filen sparas i UTF-8-format. Om detta inte fungerar för dig, så kan du alltid 
% använda dig av {\aa} för å, \''a för ä och \''o för ö.
\usepackage[utf8]{inputenc}
\usepackage[english]{babel}
% Följande väljer typsnitt som är kloner av Times New Roman, Helvetica och lämpliga till
% dem anpassade matematiktypsnitt.
\usepackage{newtxtext}
\usepackage{newtxmath}
\usepackage{qtree}
% Tabs
\newcommand\tab[1][1cm]{\hspace*{#1}}
% Links
\usepackage{hyperref}
\hypersetup{
    colorlinks,
    citecolor=black,
    filecolor=black,
    linkcolor=black,
    urlcolor=black
}

% Titlar
\usepackage{titlesec}
\usepackage{etoolbox} % https://bugs.launchpad.net/ubuntu/+source/texlive-extra/+bug/1574052
\makeatletter
\patchcmd{\ttlh@hang}{\parindent\z@}{\parindent\z@\leavevmode}{}{}
\patchcmd{\ttlh@hang}{\noindent}{}{}{}
\makeatother
\titleformat{\section}
{\fontsize{14}{15}\bfseries}{\thesection}{1em}{}
\titleformat{\subsection}
{\fontsize{10}{15}\bfseries}{\thesubsection}{1em}{}
\titleformat{\subsubsection}
{\fontsize{9}{15}\bfseries}{\thesubsubsection}{1em}{}

\usepackage[]{algorithm2e}

\usepackage{array}

\usepackage[parfill]{parskip} % Separate paragraphs by linebreak instead of indent

\usepackage{url}

\begin{document}

\textbf{Disclaimer}

This is an attempt on a summary of the keywords\footnote{\url{https://www.ida.liu.se/~TDDD08/misc/keywords.txt}} given for the course TDDD08. For your own sake, don't assume that anything is correct. The explanations have been taken from the course book\footnote{\url{http://www.ida.liu.se/~ulfni/lpp/}} and the lectures\footnote{\url{https://www.ida.liu.se/~TDDD08/lectures.shtml}}.

\tableofcontents

\section{Logic}

\subsection{Language}
In the formal language of predicate logic objects are represented by terms.

In natural language only certain combinations of words are meaningful sentences.
The counterpart of sentences in predicate logic are special constructs built from terms.
\subsubsection{Term}
A term is an element of the language,
\begin{itemize}
    \item All variables are terms
    \item All constants are terms
    \item If $f$ is an n-ary functor and $t_1,\ldots,t_n$ are terms is $f(t_1,\ldots,t_n)$ a term
\end{itemize}

\subsubsection{Functor}
\begin{itemize}
    \item A functor takes $n$ terms and evaluates to a new term
    \item A constant can be seen as a functor without arguments
\end{itemize}

\subsubsection{Predicate}
A predicate takes $n$ terms and evaluates to a truth-value
\begin{itemize}
    \item A 1-ary predicate can be seen as a property
    \item A n-ary predicate can be seen as a relation
\end{itemize}

\subsubsection{Free/bound variables}
Let $F$ be a formula and $X$ a variable. An occurance of $X$ in $F$ is said to be bound to $F$ if
\begin{itemize}
    \item The occurence follows a quantifier, or
    \item The occurence is part of a subformula that follows $\forall{X}$ or $\exists{X}$.
\end{itemize}
If the occurence of $X$ is not bound it is said to be free.

\subsubsection{Ground term/formula}
A term or formula is said to be ground if it does not contain any variables.

\subsubsection{Closed formula}
A closed formula is a formula that does not contain any free variables.

\subsubsection{Atomic formula (atom)}
An atomic formula is a logical statement (a predicate applied to a tuple of terms without logical connectives).
\begin{itemize} % https://math.stackexchange.com/questions/1110656/difference-between-terms-and-atomic-formula
  \item If $P$ is an $n$-ary predicate symbol and $t_1, t_2, \ldots, t_n$ are terms, then $P(t_1,t_2,\ldots,t_n)$ is an atomic formula.
\end{itemize}



\subsection{Interpretation}
The interpretation of constants, functors and predicate symbols provide a basis for assigning truth values to the formulas.

An interpretation $\mathcal{J}$ of an alphabet $\mathcal{A}$ is a non-empty domain $\mathcal{D}$ (or $|\mathcal{J}|$) and a mapping that associates
\begin{itemize}
    \item Each constanct $c\in\mathcal{A}$ with an element $c_\mathcal{J}\in\mathcal{D}$
    \item Each $n$-ary functor $f\in\mathcal{A}$ with a function $f_\mathcal{J}:\mathcal{D}^n\rightarrow D$
    \item Each $n$-ary predicate symbol $p\in\mathcal{A}$ with a relation $p_\mathcal{J}\subseteq\mathcal{D}\times\ldots\times\mathcal{D}$ ($n$ times)
\end{itemize}

\subsubsection{Valuation}
A valuation $\varphi$ is a mapping from variables of the alphabet to the domain of an interpretation. It is a function that assigns objects of an interpretation to the variables of the language.

Let $\mathcal{J}$ be an interpretation, $\varphi$ a valuation and $t$ a term. Then $\varphi_\mathcal{J}(t)$ is an element in $\mathcal{D}$ defined as
\begin{itemize}
    \item If $t$ is a constant $c$ then $\varphi_\mathcal{J}(t) := c_\mathcal{J}$
    \item If $t$ is a variable $X$ then $\varphi_\mathcal{J}(t) := \varphi(X)$
    \item If $t$ is of the form $f(t_1,\ldots,t_n)$ then $\varphi_\mathcal{J}(t) := f_\mathcal{J}(\varphi_\mathcal{J}(t_1),\ldots,\varphi_\mathcal{J}(t_n))$
\end{itemize}
$\varphi[X\mapsto t]$ denotes the valuation which is identical to $\varphi$ except that $X$ maps to $t$.



\subsection{Model}
An interpretation $\mathcal{J}$ is said to be a model of a set of closed formulas $P$ iff every formula $F\in P$ is true in $\mathcal{J}$.

\subsubsection{Satisfiability}
A satisfiable set of closed formulas has atleast one model.

\subsection{Logical consequence}
A closed formula $F$ is called a logical consequence of a set of closed formulas $P$ iff F is true in every model of $P$ ($P\models F$).

A way to prove $P\models F$ is to show that $\lnot F$ is false in every model of $P$. That $P\cup\{\lnot F\}$ is unsatisfiable.
\subsection{Soundness, completeness}
\begin{itemize}
    \item Inference rules are said to be sound if what is possible to derive with them ($P\vdash F$) are also logical consequences ($P\models F$).
    \item Inference rules are said to be complete if they make every logical consequences ($P\models F$) derivable ($P\vdash F$) using the rules.
\end{itemize}

\section{Definite programs}
A definitie program is a finite set of definite clauses.

\subsection{Syntax (rules, facts, queries/goals)}

\subsubsection{Definite clause}
A definite clause is a clause that contains exactly one positive literal.
\begin{itemize}
    \item $\forall(A_0\lor\lnot A_1\lor\ldots\lor\lnot A_n)$ which is equivalent to
    \item $A_0\leftarrow A_1,\ldots,A_n$
\end{itemize}

\subsubsection{Rule}
A rule is a definite clause where $n\geq1$.
\begin{itemize}
    \item $A_0\leftarrow A_1,\ldots,A_n$
\end{itemize}

\subsubsection{Fact}
A fact is a definite clause where $n = 0$.
\begin{itemize}
    \item $A_0$
\end{itemize}

\subsubsection{Queries/Goals}
Queries and goals are definite clauses with no positive literals.
\begin{itemize}
    \item $\leftarrow A_1,\ldots,A_n$
\end{itemize}
The difference between a query and a goal is that a query asks a question (contains variables) and a goal verifies a statement. A query returns solutions with respect to variables, a goal answers yes or no.

\pagebreak

\subsection{Proof trees}
A proof tree (implication tree) for a definitie program $P$ is a finite tree in which
\begin{itemize}
    \item $B_1,\ldots,B_n$ are children of a node $B$ $\Rightarrow$
    \\$B\leftarrow B_1,\ldots,B_n$ is an instance of a fact of $P$.
\end{itemize}
For any definite program $P$,atoms $A_1,\ldots,A_n$ and $A$
\begin{itemize}
    \item $P\models A_1,\ldots,A_n$ iff $P\models A_1$, and $\ldots$, and $P\models A_n$
    \item $P\models A$ iff $A$ is the root of some proof tree for $P$. 
\end{itemize}
\textbf{Example}
Let the program be


\begin{tabular}{l l}
    $c(X,Z) \leftarrow e(X,Z)$ & $e(a,b). e(d,f)$
    \\
    $c(X,Z) \leftarrow e(X,Y),c(Y,Z)$ & $e(b,d)$
\end{tabular}


Two examples of proof trees of the program are


\begin{tabular}{m{3cm} m{1cm} m{3cm}}
    Tree 1 & Tree 2\\
    \Tree [.$c(a,d)$ $e(a,b)$ [.$c(b,d)$ $e(b,d)$ ] ] & \Tree[.$c(a,f)$ $e(a,b)$ [.$c(b,f)$ $e(b,d)$ [.$c(d,f)$ $e(d,f)$ ] ] ]
    \end{tabular}


\subsection{Atomic logical consequences of a program (proof tree roots)}
The atomic logical consequences of $P$ are all the atomic formulae which may be obtained from the facts of $P$ by applying modus ponens.

For sets $J$ of possible non-ground atoms
\begin{itemize}
    \item $T_P^c(J)=\{A\mid A\leftarrow A_1,\ldots,A_n\textmd{ is an instance of a }C\in P, \{A_1,\ldots,A_n\}\subseteq J\}$
    \begin{itemize}
        \item  $(T^c_P)^i(\emptyset)$ denotes the roots of proof trees of height less than $i$.
    \end{itemize}
\end{itemize}
Let $PTR(P)$ be the \textbf{proof tree roots} of $P$. Then
\begin{itemize}
    \item $PTR(P) = \bigcup\limits_{i=1}^\infty(T^c_P)^i(\emptyset)$
\end{itemize}

\subsection{Correctness / completeness (of a program) w.r.t. a specification}

\subsection{Incorrectness diagnosis, incompleteness diagnosis}

\pagebreak

\subsection{Herbrand universe/base/interpretation}
Let $\mathcal{A}$ be an alphabet with at least one constant symbol
\begin{itemize}
  \item The \textbf{Herbrand universe of $\mathcal{A}$} ($U_\mathcal{A}$) is the set of all ground terms constructed from the alphabets functors and constants.
  \item The \textbf{Herbrand base of $\mathcal{A}$} ($B_\mathcal{A}$) is the set of all ground, atomic formulas over the alphabet [all the predicates and every combination of the elements in the universe as input].
\end{itemize}
The Herbrand universe and base are (for our purposes) defined for a given program $P$. The alphabet is assumed to contain the symbols that appear in $P$.
\begin{itemize}
  \item The \textbf{Herbrand interpretation of $P$} is an interpretation $\mathcal{J}$ that
  \begin{itemize}
    \item The domain of $\mathcal{J}$ is $U_p$
    \item For every constant $c$, $c_\mathcal{J}$ is defined as $c$
    \item For every $n$-ary functor $f$ the function $f_\mathcal{J}$ is defined as \\$f_\mathcal{J}(x_1,\ldots,x_n):=f(x_1,\ldots,x_n)$
    \item For every $n$-ary predicate symbol $p$, $p_\mathcal{J}$ is a subset of $U^n_P$
  \end{itemize}
\end{itemize}
The Herbrand interpretation has predefined meanings of functors and constants. In order to specify a Herbrand interpretation it suffices to list the relations associated with the predicate symbol.

\textbf{Example} For an $n$-ary predicate symbol $p$ and a Herbrand interpretation $\mathcal{J}$ the meaning of $p_\mathcal{J}$ consists of the set $\{(t_1,\ldots,t_n)\in U^n_P\mid \mathcal{J}\models p(t_1,\ldots,t_n)\}$
\subsection{Least Herbrand model}
\begin{itemize}
  \item A \textbf{Herbrand model of $P$} is a Herbrand interpretation $\mathcal{J}$ which is a model of every formula $F\in P$ $(\mathcal{J}\models F)$.
  \item The \textbf{least Herbrand Model of $P$} ($M_P$) is the set of all ground atomic logical consequences of the program ($M_P = \{A\in B_P \mid P\models A\}$).
\end{itemize}
\subsection{Immediate consequence operator}
\section{SLD-resolution}

\subsection{Substitutions}
A substitution is a finite set of pairs of terms $\{X_1/t_1,\ldots,X_n/t_n\}$ where
\begin{itemize}
    \item $t_i$ is a term
    \item $X_i$ is a variable
    \item $X_i\not=t_i$ and $X_i\not=X_j$
\end{itemize}
The \textbf{empty substitution} is denoted $\epsilon$.

\subsubsection{Application}

$X\theta$ denotes the application of a substitution $\theta$ to a variable $X$.

$X\theta=\begin{cases}
    t; X/t\in\theta\\
    X; \textmd{else}
\end{cases}
$

\subsubsection{Composition}
Let $\theta$ and $\sigma$ be two substitutions
\begin{itemize}
    \item $\theta=\{X_1/s_1,\ldots,X_n/s_n\}$
    \item $\sigma=\{Y_1/t_1,\ldots,Y_n/t_n\}$
\end{itemize}
The composition $\theta\sigma$ is obtained from $\{X_1/s_1\sigma,\ldots,X_m/s_m\sigma, Y_1/t_1,\ldots,Y_n/t_n\}$ by
\begin{itemize}
    \item Removing all $X_i/s_i\sigma$ where $X_i=s_i\sigma$
    \item Removing all $Y_j/t_j$ for which $Y_j\in\{X_1,\ldots,X_m\}$
\end{itemize}

\subsection{Unifier}

A unifier is a substitution $\theta$ such that applying $\theta$ to terms $s, t$ makes them identical ($\theta s=\theta t$). $\theta$ is a unifier of $s, t$. The search for a unifier is the process of solving $s \doteq t$.


\subsubsection{Generality of substitutions}
A substitution $\theta$ is more general than a substitution $\sigma$ ($sigma\preceq\theta$) iff there exists a substitution $\omega$ such that $\sigma=\theta\omega$

\subsubsection{Most general unifier}
A unifier $\theta$ is said to be a mgu of two terms iff $\theta$ is more general than any other unifier of the terms.

\subsubsection{Solved form}
A set of equations $\{X_1\doteq t_1,\ldots,X_n\doteq t_n\}$ is said to be in solved form iff $X_1,\ldots,X_n$ are distinct variables none of which appear in $t_1,\ldots,t_n$.

If  $\{X_1\doteq t_1,\ldots,X_n\doteq t_n\}$ is in solved form then $\theta=\{X_1/t_1,\ldots,X_n/t_n\}$ is an mgu of $(X_1,\ldots,X_n)$ and $(t_1,\ldots,t_n)$.

\subsection{Unification algorithm}
\begin{itemize}
    \item \textbf{Input} A set of $\varepsilon$ equations
    \item \textbf{Output} An equivalent set of equations in solved form or failure
    \item \textbf{Repeat until no action is possible on any equation in $\varepsilon$}
    \begin{itemize}
        \item Select an arbitrary $s\doteq t\in\varepsilon$
        \item \textbf{(Case 1)} $f(s_1,\ldots,s_n)\doteq f(t_1,\ldots,t_n), n\geq0$
        \begin{itemize}
            \item Replace equation by $s_1\doteq t_1,\ldots,s_n\doteq t_n$
        \end{itemize}
        \item \textbf{(Case 2)} $f(s_1,\ldots,s_n)\doteq g(t_1,\ldots,t_n), f/m\not=g/n$
        \begin{itemize}
            \item Halt with failure
        \end{itemize}
        \item \textbf{(Case 3)} $X\doteq X$
        \begin{itemize}
            \item Remove the equation
        \end{itemize}
        \item \textbf{(Case 4)} $t\doteq X$, $t$ is not a variable
        \begin{itemize}
            \item Replace the equation by $X\doteq t$
        \end{itemize}
        \item \textbf{(Case 5)} $X\doteq t$, $X\not=t$, $X$ occurs more than once in $\varepsilon$
        \begin{itemize}
            \item \textbf{(Case 5a)} $X$ is a proper subterm of $t$
            \begin{itemize}
                \item Halt with failure
            \end{itemize}
            \item \textbf{(Case 5b)} 
            \begin{itemize}
                \item Replace all other occurences of $X$ by $t$
            \end{itemize}
        \end{itemize}
    \end{itemize}
\end{itemize}
Due to inefficiency of occurs check (Case 5a) it is often removed.

In Prolog this check is omitted which means that Case 5b is always executed when $X\doteq f(X)$. An infinite tree is constructed, $X\doteq f(f(...))$ (actually a graph with a cycle is built).

\subsubsection{Renaming}
A substitution $\{X_1/Y_1,\ldots,X_n/Y_n\}$ is called a renaming substitution iff $Y_1,\ldots,Y_n$ is a permutation of $X_1,\ldots,X_n$

If $\theta=\{X_1/Y_1,\ldots,X_n/Y_n\}$ is a renaming then $\theta^{-1}=\{Y_1/X_1,\ldots,Y_n/X_n\}$ is its reverse. $\theta\theta^{-1}=\epsilon$

\subsection{SLD-resolution}
Given a program $P$, compute answers for an initial query $Q_0$.
The computation step is done by
\begin{itemize}
    \item $Q = A_1,\ldots,\boldsymbol{A_i},\ldots,A_m$ - a query
    \item $C = \boldsymbol{H}\leftarrow B_i,\ldots,B_n$ - a variant of a clause of $P$
    \item $\theta$ - mgu of $\boldsymbol{A_i}$ and $\boldsymbol{H}$
    \\
    \item $Q^\prime=(A_i,\ldots,A_{i-1},\boldsymbol{B_1,\ldots,B_n},A_{i+1},\ldots,A_m)\theta$
\end{itemize}
Note that $P\models Q^\prime \rightarrow Q\theta$

\subsection{SLD-derivation}
An SLD-derivation is a sequence of queries, clauses and mgu:s
\begin{itemize}
    \item $Q_0-_{\theta_1}^{C_1}\rightarrow Q_1-_{\theta_2}^{C_2}\rightarrow Q_2\ldots Q_{n-1}-_{\theta_n}^{C_n}\rightarrow Q_n$
\end{itemize}
Where
\begin{itemize}
    \item Each $C_i$ is a variant of a clause of $P$
    \item Each $Q_i$ is a resolvent of $Q_{i-1}$ and $C_i$ with mgu $\theta_i$
    \item No variables of $C_i$ occurs in $Q_0,\ldots,Q_{i-1}$, $\theta_1,\ldots,\theta_{i-1}$ and $C_1,\ldots,C_{i-1}$ 
    \\(the variables in $C_i$ are new)
\end{itemize}
Obtaining a clause variant satisfying the last requirement requires renaming.

Note that $P\models Q_n\rightarrow Q_0\theta_1\ldots\theta_n$. A derivation is successful if $Q_n$ is true, then $P\models Q_0\theta_1\ldots\theta_n$. $P\models Q_0\theta_1\ldots\theta_n$ is the computated answer for $P$ and $Q_0$.

An SLD-resolution is sound since each computed answer is a correct answer. It can be seen as an attempt to construct a proof tree for some instance of $A_i$ top-down.
\begin{itemize}
    \item $Q_j$ leaves - a partially constructed tree
    \item Other leaves - instances of facts of the program
    \item $A\theta_1\ldots\theta_j$ - the root of the tree
    \item $Q_0-_{\theta_1}^{C_1}\rightarrow Q_1-_{\theta_2}^{C_2}\rightarrow Q_2\ldots-_{\theta_n}^{C_n}\rightarrow \blacksquare$ - a successful derivaton
\end{itemize}

\pagebreak

\textbf{Example}

Let the program $P$ be

\begin{tabular}{l l}
    $c(X,Z) \leftarrow e(X,Z)$ & $e(a,b). e(d,f)$
    \\
    $c(X,Z) \leftarrow e(X,Y),c(Y,Z)$ & $e(b,d)$
\end{tabular}

Then

\begin{tabular}{l | l}
    $Q_0=c(X,Y)$ & $C_1=c(X^\prime, Z^\prime)\leftarrow e(X^\prime,Y^\prime),c(Y^\prime,Z^\prime)$
    \\
    & $\theta_1=\{X^\prime/X, Z^\prime/Y\}$
    \\\hline
    $Q_1=e(X,Y^\prime), c(Y^\prime,Y)$ & $C_2=e(a,b)$
    \\
    & $\theta_2=\{X/a, Y^\prime/b\}$
    \\\hline
    $Q_2=c(b,Y)$ & $C_3=c(X^{\prime\prime}, Z^{\prime\prime})\leftarrow e(X^{\prime\prime},Y^{\prime\prime})$
    \\
    & $\theta_3=\{X^{\prime\prime}/b, Z^{\prime\prime}/Y\}$
    \\\hline
    $Q_3=c(X,Y)$ & $C_4=e(b,d)$
    \\
    & $\theta_4=\{Y/d\}$
    \\\hline
    $Q_5=\blacksquare$
\end{tabular}
\subsection{SLD-tree}
For an SLD-tree for program $P$, query $Q$ via (computed with) rule $\mathcal{R}$ the following is true
\begin{itemize}
    \item Branches are SLD-derivations for $P, Q$ via $\mathcal{R}$
    \item A node $Q^\prime$ with $A$ selected has exactly one child for each applicable clause $C$ of $P$. The child is a resolvent of $Q^\prime$ and a variant of $C$ with respect to $A$
\end{itemize}
\subsection{Selection rule (computation rule)}
A function selecting atom in a query $Q_i$ to be resolved is called a selection rule or a computation rule. The Prolog selection rule selects the first atom of the query to be resolved.


\subsection{Soundness and completeness}

\section{Negation}
\subsection{Closed world assumption}
\subsection{Negation as (finite) failure}
\subsection{Program completion}
\subsection{General logic programs}
\subsection{SLDNF-resolution, SLDNF-forest}
\subsubsection{successful derivation, failed tree, floundering}

\section{Definite clause grammars}
\subsection{Translation to Prolog}

\section{Prolog}
\subsection{Terms as the data structures of logic programming}
\subsection{Recursive data structures}
\subsection{Lists and difference lists}
\subsection{Constraints (especially FD-constraints)}
\subsection{The cut - don't use it.}

\end{document}