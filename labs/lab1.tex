\section{Lab 1}
\subsection{Exercise 1.1}
\lstinputlisting[language=Prolog]{./1/ex1.pl}
\subsubsection{Queries}
\textbf{Who is happy?}
\begin{lstlisting}
| ?- happy(X).
X = nisse ? ;
X = bettan ? ;
no
\end{lstlisting}
\textbf{Who likes who?}
\begin{lstlisting}[language=Prolog]
| ?- likes(X, Y).
X = nisse,
Y = ulrika ? ;
X = peter,
Y = ulrika ? ;
X = bosse,
Y = ulrika ? ;
no
\end{lstlisting}
\textbf{How many persons like Ulrika?}
\begin{lstlisting}[language=Prolog]
| ?- findall(X, likes(X, ulrika), Z), length(Z, N).
Z = [nisse,peter,bosse],
N = 3 ? ;
no
\end{lstlisting}
\pagebreak
\subsubsection{Questions}
\begin{itemize}
  \item In what way should you arrange the clauses (rules and facts) in the program?
  \begin{itemize}
    \item Prolog will execute rules or facts with same name from top to down which means placing a fact before a rule is more efficient. This is because the fact gives a truth value while the truth value of a rule depends on other rules or facts.

    Readability is increased by grouping same named rules and facts together. 
  \end{itemize}
  \item In what way should the premises of rules be arranged?
  \begin{itemize}
    \item Prolog will execute the premises of a rule from left to right. This means that facts should precede rules because verifying a rule would lead to a new node in the SLD-tree even if some fact of the group of premises is false. If a fact is refuted the branch fails and processing stops. Same reasoning could be applied to the ordering of rules, computationally simple rules should generally precede computationally complex ones.
  \end{itemize}
\end{itemize}

\pagebreak

\subsection{Exercise 1.2}
\lstinputlisting[language=Prolog]{./1/ex2.pl}
\subsubsection{Queries}
\begin{lstlisting}[language=Prolog]
| ?- npath(a, g, L).
L = 5 ? ;
L = 4 ? ;
L = 5 ? ;
L = 4 ? ;
L = 3 ? ;
L = 4 ? ;
no
\end{lstlisting} 